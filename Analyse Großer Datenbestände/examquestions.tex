%%%%%%%%%%%%%%%%%%%%%%%%%%%%%%%%%%%%%%%%%
% Short Sectioned Assignment
% LaTeX Template
% Version 1.0 (5/5/12)
%
% This template has been downloaded from:
% http://www.LaTeXTemplates.com
%
% Original author:
% Frits Wenneker (http://www.howtotex.com)
%
% License:
% CC BY-NC-SA 3.0 (http://creativecommons.org/licenses/by-nc-sa/3.0/)
%
%%%%%%%%%%%%%%%%%%%%%%%%%%%%%%%%%%%%%%%%%

%----------------------------------------------------------------------------------------
%	PACKAGES AND OTHER DOCUMENT CONFIGURATIONS
%----------------------------------------------------------------------------------------

\documentclass[paper=a4, fontsize=11pt]{scrartcl} % A4 paper and 11pt font size

\usepackage[T1]{fontenc} % Use 8-bit encoding that has 256 glyphs
\usepackage[ngerman]{babel}
\usepackage{fourier} % Use the Adobe Utopia font for the document - comment this line to return to the LaTeX default
\usepackage{amsmath,amsfonts,amsthm} % Math packages
\usepackage{graphicx}
\usepackage[utf8]{inputenc}
\usepackage{listings}
\usepackage[section]{placeins}
\usepackage{lipsum} % Used for inserting dummy 'Lorem ipsum' text into the template
\usepackage{float}
\usepackage{multicol}
\usepackage{algpseudocode}
\usepackage{algorithm}% http://ctan.org/pkg/algorithms
\usepackage{graphicx,wrapfig}

% Algorithmic modifications
\makeatletter
\newcommand{\ALOOP}[1]{\ALC@it\algorithmicloop\ #1%
  \begin{ALC@loop}}
\newcommand{\ENDALOOP}{\end{ALC@loop}\ALC@it\algorithmicendloop}
\renewcommand{\algorithmicrequire}{\textbf{Input:}}
\renewcommand{\algorithmicensure}{\textbf{Output:}}
\newcommand{\algorithmicbreak}{\textbf{break}}
\newcommand{\BREAK}{\STATE \algorithmicbreak}
\makeatother

\usepackage{sectsty} % Allows customizing section commands
\allsectionsfont{\centering \normalfont\scshape} % Make all sections centered, the default font and small caps

\usepackage{fancyhdr} % Custom headers and footers
\pagestyle{fancyplain} % Makes all pages in the document conform to the custom headers and footers
\fancyhead{} % No page header - if you want one, create it in the same way as the footers below
\fancyfoot[L]{} % Empty left footer
\fancyfoot[C]{} % Empty center footer
\fancyfoot[R]{\thepage} % Page numbering for right footer
\renewcommand{\headrulewidth}{0pt} % Remove header underlines
\renewcommand{\footrulewidth}{0pt} % Remove footer underlines
\setlength{\headheight}{13.6pt} % Customize the height of the header

\usepackage{enumitem}
\setlistdepth{20}
\renewlist{itemize}{itemize}{20}

% initially, use dots for all levels
\setlist[itemize]{label=$\cdot$}

% customize the first 3 levels
\setlist[itemize,1]{label=\textbullet}
\setlist[itemize,2]{label=--}
\setlist[itemize,3]{label=*}

\numberwithin{equation}{section} % Number equations within sections (i.e. 1.1, 1.2, 2.1, 2.2 instead of 1, 2, 3, 4)
\numberwithin{figure}{section} % Number figures within sections (i.e. 1.1, 1.2, 2.1, 2.2 instead of 1, 2, 3, 4)
\numberwithin{table}{section} % Number tables within sections (i.e. 1.1, 1.2, 2.1, 2.2 instead of 1, 2, 3, 4)

\setlength\parindent{0pt} % Removes all indentation from paragraphs - comment this line for an assignment with lots of text

\DeclareMathOperator*{\argmin}{arg\,min}

%----------------------------------------------------------------------------------------
%	TITLE SECTION
%----------------------------------------------------------------------------------------

\newcommand{\horrule}[1]{\rule{\linewidth}{#1}} % Create horizontal rule command with 1 argument of height

\title{
\normalfont \normalsize
%\textsc{Karlsruher Insitut für Technologie} \\ [25pt] % Your university, school and/or department name(s)
\horrule{0.5pt} \\[0.4cm] % Thin top horizontal rule
\huge Softwaretechnik II\\ Zusammenfassung WS17/18 % The assignment title
\horrule{2pt} \\[0.5cm] % Thick bottom horizontal rule
}

\author{Manuel Lang} % Your name

\date{} % Today's date or a custom date

\begin{document}

\begin{itemize}
\item Was ist Overfitting?
\item Was ist Change Detection? Wie unterschieden sich multivariate von univariaten Datenströmen?
\item ‚Kategorisierung von Daten‘ (erste Folie dieses Kapitels) sprechend wiedergeben können.
\item Gegeben Aggregationsfunktion X, ist sie distributiv/algebraisch/holistisch/self-maintainable?
\item Sehen Sie einen allgemeinen Zusammenhang zwischen distributiv/algebraisch/holistisch auf der einen und self-maintainable auf der anderen Seite? (Es gibt einen.)
\item Wie groß ist die Entropie, wenn alle Klassen gleich häufig sind? Schreiben Sie dafür eine allgemeine Formel hin.
\item Welche Möglichkeiten kennen Sie, die Stärke des Zusammenhangs zwischen zwei Zufallsvariablen zu quantifizieren?
\item  Welche statistischen Tests kennen Sie?
\item Wofür genau sind die einzelnen Tests gut?
\item Was bedeutet Dimensionality Reduction?
\item Welche Möglichkeiten der Dimensionality Reduction kennen Sie?
\item Welche Diskretisierungsverfahren kennen Sie?
\item Wie findet man jeweils den besten Merge bzw. Split?
\item Warum kann man für räumliche Anfragen nicht ohne weiteres auswerten, wenn man für jede Dimension separat einen B-Baum angelegt hat?
\item Wie ist der R-Baum aufgebaut?
\item Wie funktioniert die Suche nach dem nächsten Nachbarn mit dem R-Baum? (Frage ich ganz gerne u. a. dann, wenn die Prüfung stockend verläuft...)
\item Was ändert sich, wenn die Objekte eine räumliche Ausdehnung haben? Dto. Anfragen.
\item Stören uns Überlappungen von Knoten des R-Baums? Wenn ja, warum?
\item Wie unterscheiden sich R-Baum, kD-Baum und kDB-Baum? (Wie Balancierung für kDB-Baum funktioniert, muss man für Prüfung nicht wissen.)
\item Wie funktioniert Einfügen in den R-Baum, inklusive Split?
\item Wie baut man einen Entscheidungsbaum auf?
\item Wie kann man Overfitting beim Aufbau eines Entscheidungsbaums berücksichtigen?
\item Wie kann Aufbau des Entscheidungsbaums berücksichtigen, dass unterschiedliche Fehlerarten unterschiedlich schlimm sind?
\item Was ist die „10-fold cross validation“?
\item Wie haben wir die Erfolgsquote definiert?
\item Was ist ein Lift Chart? Wie unterscheidet es sich von der ROC Kurve?
\item Was für Fehlerarten gibt es bei Vorhersagen von Klassenzugehörigkeiten?
\item Was für Kennzahlen kennen Sie, die diese Fehlerarten sämtlich berücksichtigen?
\item Was ist Unterschied zwischen Kovarianz und Correlation Coefficient?
\item Warum kommt bei der informational loss Funktion die Logarithmusfunktion zur Anwendung?
\item Was sind Association Rules?
\item Wie findet man sie?
\item Wie überprüft man rasch für viele Transaktionen, welche Kandidaten sie enthalten?
\item Geben Sie ein Beispiel für eine Association Rule mit hohem/niedrigem Support und hoher/niedriger Confidence.
\item Was sind multidimensionale Association Rules?
\item Was sind Multi-Level Association Rules, und wie findet man sie?
\item In welchen Situationen ist Apriori teuer, und warum?
\item Was kann man gegen diese Schwächen tun?
\item Was sind FP-Trees, und wie lassen sie sich für die Suche nach Frequent Itemsets verwenden?
\item Was kann man tun, wenn FP-Trees für den Hauptspeicher zu groß sind?
\item Was ist Constraint-basiertes Mining? Was sind die Vorteile?
\item Was für Arten von Constraints kennen sie? Beispiele hierfür.
\item Was ist Anti-Monotonizität, Succinctness? <Für ein bestimmtes Constraint sagen/begründen, ob anti-monoton/succinct.>
\item Wie lässt sich Apriori für das Mining von Teilfolgen verallgemeinern?
\item Antagonismus von Support-basiertem und Constraint-basiertem Pruning erklären können.
\item Alternativen für Constraint-basiertes Pruning (wenn Constraint nicht anti-monoton) erklären können.
\item Welche Clustering-Verfahren kennen Sie?
\item Gegeben Szenario X, welche Clustering-Verfahren sind sinnvoll, und warum?
\item Erklären Sie Clustering-Verfahren XY.
\item Warum funktionieren herkömmliche Clustering-Verfahren in hochdimensionalen Merkmalsräumen nicht?
\item Skizzieren Sie eine mögliche Lösung.
\item Erklären Sie, warum Clustering mit kategorischen Attributen besonders ist? Warum ist Link-basiertes Clustering hier hilfreich?
\item Welche Definitionen für Outlier kennen Sie?
\item Gegeben die abstandsbasierte Definition von Outlier, welche Techniken zur Ermittlung der Outlier kennen Sie?
\item Warum kann man beim Dichte-basierten Clustering nicht einfach die Dichte um die Punkte herum vergleichen und die mit der geringsten Dichte zurückgeben?
\item Sehen Sie einen Zusammenhang zwischen Clustering und Outlier Detection?
\item Welche Anomalien hochdimensionaler Merkmalsräume kennen Sie?
\item Wieso funktionieren hierarchische Indexstrukturen in hochdimensionalen Merkmalsräumen nicht?
\item Was ist der Zusammenhang zwischen der Zelldichte und Outlier Detection in hochdimensionalen Merkmalsräumen?
\item Wie groß sollten die Zellen sein?
\item Geben Sie die Klassifizierung aus der LV in 'interessante' und 'weniger interessante' Outlier wieder.
\item What do we mean with ‚physically consistent models‘?
\item What is matrix completion?
\item What are residuals?
\item Describe the objective and the main ingredients of GLUE.
\item Please reproduce the classification of approaches for theory-guided data science from this chapter.
\end{itemize}

\end{document}
